
\documentclass{beamer}
\beamertemplatenavigationsymbolsempty
\usecolortheme{beaver}
\setbeamertemplate{blocks}[rounded=true, shadow=true]
\setbeamertemplate{footline}[page number]
%
\usepackage{adjustbox}
\usepackage[utf8]{inputenc}
\usepackage[english]{babel}
\usepackage{amssymb,amsfonts,amsmath,mathtext}
\usepackage{subfig}
\usepackage[all]{xy}
\usepackage{array}
\usepackage{multicol}
\usepackage{hyperref}
\usepackage{hhline}
\usepackage{algorithm}
\usepackage{algorithmic}
\usepackage{algpseudocode}
\usepackage{tabularx}
\graphicspath{ {fig/} {../fig/} }

%----------------------------------------------------------------------------------------------------------
\title[\hbox to 56mm{Bayesian Causal Inference in the Convergent Cross-Mapping test and Canonical Correlation Analysis}
\author[Placeholder Authors]{Bayesian Causal Inference in the Convergent Cross-Mapping test and Canonical Correlation Analysis}
\institute{Moscow Institute of physics and Technology}
\date{\footnotesize

\par\smallskip\emph{Author:} Mark Ikonnikov \
\par\smallskip\emph{Expert:} Vadim Strijov \\
\par\bigskip\small December, 2025}

%----------------------------------------------------------------------------------------------------------
\begin{document}
%----------------------------------------------------------------------------------------------------------
\begin{frame}
\thispagestyle{empty}
\maketitle
\end{frame}

\begin{frame}{Research Objective: Bayesian Causal Inference for CCA/CCM}
\begin{block}{Objectives}
\begin{enumerate}
    \item Formulate probabilistic CCA with Bayesian interpretation.
    \item Introduce \texttt{do(X)} intervention operator for causal inference for CCA.
    \item Formulate the Bayesian Causal Inference with \texttt{do(X)} intervention operator to be used in the Convergent Cross-Mapping test (TO BE DONE)
    \item Validate via d-variate time series, text document. (TO BE DONE)
\end{enumerate}
\end{block}
\end{frame}

\begin{frame}{Motivation: Why Causal Inference in Time Series?}
\begin{columns}[c]
\column{0.7\textwidth}
    \includegraphics[width=1.0\textwidth]{cca_ci.jpg} \\
    \textit{Understanding causal influence between coupled series via CCA.}
\column{0.5\textwidth}
    \begin{itemize}
        \item Classical CCA: captures correlations, not causation.
        \item CCM: tests predictive causation via shadow manifolds.
        \item Bayesian CCA + \texttt{do(X)}: allows reasoning about interventions.
    \end{itemize}
\end{columns}
\end{frame}



\begin{frame}{Background and Literature}
\begin{tabularx}{\textwidth}{>{\raggedright\arraybackslash}X
                                c
                                >{\raggedright\arraybackslash}p{3cm}
                                >{\raggedright\arraybackslash}p{2cm}}
\toprule
\textbf{Title} & \textbf{Year} & \textbf{Authors} & \textbf{Paper} \\
\midrule
A Probabilistic Interpretation of Canonical Correlation Analysis
& 2001
& Bach et al.
& \href{https://www.jmlr.org/papers/volume3/bach02a/bach02a.pdf}{JMLR} \\

Canonical Correlation Analysis: An Overview with Application to Learning Methods
& 2012
& Hardoon et al.
& Neural Computation \\

Learning Relationships between Text, Audio, and Video via Deep CCA for Multimodal Language Analysis
& 2020
& Sun et al.
& AAAI \\
\bottomrule
\end{tabularx}
\end{frame}


%------------------------------------------------
\begin{frame}{Probabilistic Model of CCA}
\begin{theorem}[Probabilistic CCA]
Let $Z \sim \mathcal{N}(0,I_k)$, then
\[
X = AZ + \varepsilon_X, \quad \varepsilon_X \sim \mathcal{N}(0,\Psi_X), \quad
Y = BZ + \varepsilon_Y, \quad \varepsilon_Y \sim \mathcal{N}(0,\Psi_Y).
\]
There exists a parametrization $(A^*,B^*)$ such that the classical canonical directions $(U,V)$ satisfy
\[
A^* = C_{11}^{1/2} U \Lambda^{1/2}, \quad B^* = C_{22}^{1/2} V \Lambda^{1/2}.
\]
\end{theorem}
\begin{block}{Insight}
pCCA represents CCA as a latent-variable model and relates probabilistic parameters to canonical directions.
\end{block}
\end{frame}

%------------------------------------------------
\begin{frame}{SCM with $\text{do}(X)$ Intervention}

Consider the model
\[
Z \sim \mathcal N(0,I_k), \quad 
X = AZ + U_X, \quad 
Y = CX + BZ + U_Y.
\]
Then $\mathcal M = \langle \mathcal U, \mathcal V, \mathcal F, P(\mathcal U)\rangle$ defines a structural causal model for $X \to Y$ with hidden confounder $Z$ and direct effect $C$.
\begin{block}{Insight}
Formalizes causal relationships and sets the basis for interventional analysis via \texttt{do(X)}.
\end{block}
\end{frame}

%------------------------------------------------
\begin{frame}{Interventional Distribution}
\begin{theorem}[Interventional Distribution]
For the SCM above,
\[
P(Y \mid do(X=x)) = \mathcal N(Cx, BB^\top + \Psi_Y).
\]
\end{theorem}
\begin{block}{Insight}
Mean encodes direct causal effect, variance accounts for latent confounders. Allows separation of causation from correlation.
\end{block}
\begin{corollary}[Linear Causal Effect]
\[
\mathbb E[Y \mid do(X=x_1)] - \mathbb E[Y \mid do(X=x_0)] = C (x_1 - x_0)
\]
\end{corollary}
\begin{block}{Insight}
Identifies linear, pure causal effect, independent of hidden confounders.
\end{block}

\end{frame}

\begin{frame}{Projection onto CCA Subspace}

\begin{theorem}[Projection of Causal Effect]
Let $(u_i,v_i)$ be canonical directions normalized by
$u_i^\top C_{11} u_i = v_i^\top C_{22} v_i = 1$. Then
\[
v_i^\top \mathbb E[Y \mid do(X=x)] = (v_i^\top C u_i) (u_i^\top x).
\]
\end{theorem}

\begin{block}{Insight}
Shows how interventional effect projects linearly onto canonical coordinates, facilitating interpretation in CCA space.
\end{block}

\begin{theorem}[Non-Causal Nature of Canonical Correlations]
For SCM with $X = AZ$, $Y = CX + BZ$ and $\Psi_X = \Psi_Y = 0$, the canonical correlation $\rho_i$ satisfies
\[
\rho_i = u_i^\top C_{11} C^\top v_i + u_i^\top A B^\top v_i.
\]
\end{theorem}

\end{frame}

%------------------------------------------------

%------------------------------------------------
\begin{frame}{Interpretation for Time Series and CCM}
\begin{block}{Time Series}
- $\text{do}(X_t = x)$: force state at time $t$, observe $Y_{t+\tau}$.  
- SCM + pCCA isolates direct causal influence from latent dynamics.
\end{block}
\begin{block}{CCM Perspective}
- Project interventional response on reconstructed manifold.  
- Enables estimation of directed influence $X \to Y$ beyond correlation.
\end{block}
\begin{block}{Textual Analogy}
- $\text{do}(X=x)$: modify embeddings of linguistic features.  
- $Z$: latent semantic concepts.  
- Measures causal impact on related documents.
\end{block}
\end{frame}

%------------------------------------------------
\begin{frame}{Summary of Contributions}
\begin{itemize}
    \item Formulated SCM for pCCA with $\text{do}(X)$.  
    \item Derived interventional distribution and projection onto CCA space.  
    \item Demonstrated difference between canonical correlation and causal effect.  
    \item Laid foundation for CCM-based causal estimation in time series and text.
\end{itemize}
\end{frame}

\begin{frame}{Empirical Validation of the Causal pCCA Model}
\centering
\includegraphics[width=0.7\textwidth]{empirical_vs_theor.png}

\vspace{0.3cm}
\small
Comparison between the empirical response of the system and the theoretical
interventional expectation predicted by the causal pCCA model.
\end{frame}

\begin{frame}{Conclusions and Novelty}
\begin{block}{Key Findings}
\begin{itemize}
    \item Developed Probabilistic CCA with \texttt{do(X)} operator for causal inference in time series.
    \item Derived closed-form posterior distributions under intervention.
    \item Empirical results validate the theoretical formulation
\end{itemize}
\end{block}
\begin{block}{Novelty / Contribution}
\begin{itemize}
    \item \textbf{Proposed} a Probabilistic causal formulation of Canonical Correlation Analysis (CCA),
    \item \textbf{distinct from} classical correlation-based CCA by explicitly introducing a structural causal model and the $\mathrm{do}(X)$ intervention operator,
    \item \textbf{enabling} identification and analysis of causal effects in latent-variable settings.
\end{itemize}
\end{block}
\end{frame}

\end{document}